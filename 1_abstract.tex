%\section*{Constitution du jury de th\`ese:}
\afterpage{\blankpage}
\thispagestyle{empty}
\begin{Large}
\textbf{Doctoral Jury:}
\end{Large}
\begin{itemize}
\item{Prof. Chengping Shen   (Beihang Univeristy, Beijing)              , co-supervisor}
\item{Prof. Barbara Clerbaux (Universit\'e Libre de Bruxelles, Brussels), co-supervisor}
\end{itemize}

%\bigskip
%\begin{Large}
%\Large \textbf{Evaluateurs externes:}
%\end{Large}
%\begin{itemize}
%\item{Prof. Attilio Andreazza (UniMi, Milano)}
%\item{Prof. Luigi Rolandi (SNS, Pisa)}
%\end{itemize}

\newpage
\selectlanguage{english}
\section*{Abstract}
This thesis describes searches for new massive resonances that decay in an electron-positron or photon pair in the final state.
Different datasets, coming from proton-proton collisions at a center-of-mass energy $\sqrt{s}=13$ TeV at the Large Hadron Collider (LHC) and collected by the CMS experiment in 2015 and 2016, have been analyzed.
After a chapter devoted to the description of the standard model of elementary particle physics, the motivation for the introduction of new theories that go beyond the standard model are introduced and some classes of models are described.
The techniques used in order to reconstruct the particles produced in the collisions are discussed afterwards, with a special emphasis on electron/positron and
photon reconstructions. Two separate analyses are presented.
\\The first one is the search for new heavy resonances decaying in an electron-positron pair in the final state.
Such resonances are predicted by a variety of models such as grand unified theories or theories that introduce extra space-like dimensions. Their signature would appear as
a localised excess of events in the electron-positron invariant mass spectrum.
The event selection is optimized in order to be highly efficient for high-energy electrons/positrons and to avoid loosing potential signal events.
The analysis relies on simulated samples for the estimation of the main source of background, which is the standard model Drell-Yan process.
Data-driven approaches are pursued for both validating the simulation of the
subleading background processes with prompt electrons in the final state and the determination of the background coming from processes of quantum chromo dynamics.
After having inspected the electron-positron invariant mass, no excess over the standard model
expectation is observed, and 95\% confidence
level upper limits are set on the ratio of production cross-section times branching ratio of a
new resonance to the one at the Z boson peak, using the data collected in 2016 (35.9 \fbinv).
\\The second analysis presented in this thesis is the search for new heavy resonances decaying in the diphoton final state, the existence of which is predicted by models with non-minimal scalar sectors or by theories postulating the existence of additional space-like dimensions.
Their signature would appear as a localised excess of events in the diphoton invariant mass spectrum.
As for the case of the dielectron analysis, the event selection has been optimized in order to be highly efficient for high-energy photons.
The background estimation is completely data-driven and achieved via a parametrization of the observed diphoton invariant mass spectrum.
After the inspection of the diphoton invariant mass, no excess over the standard model expectation is observed, and 95\% confidence
level upper limits are set on the production cross-section times branching ratio, using the data collected in the first half of 2016 (12.9 \fbinv). Results have also been combined with those obtained with the same analysis techniques but with different datasets collected in 2012 and 2015 by the CMS experiment.

\newpage
\selectlanguage{frenchb}
\section*{Résumé}
Cette thèse décrit les recherches de nouvelles résonances massives se décomposant en une paire électron-positron ou une paire de photons dans l'état final.
Différents ensembles de données, provenant des collisions proton-proton à une énergie dans le centre de masse $\sqrt{s} = 13$ TeV
au Large Hadron Collider (LHC) et enregistrés par l'expérience CMS en 2015 et 2016, ont été analysés.
Après un chapitre consacré à la description du modèle standard de la physique des particules élémentaires,
les motivations pour postuler de nouvelles théories qui dépassent le modèle standard sont introduites
et certaines classes de modèles sont décrites. Les techniques utilisées pour reconstituer les particules
produites dans les collisions sont discutées ensuite, avec un accent particulier sur la reconstruction des électrons/positrons et des photons. Deux analyses séparées sont présentées.
\\La première est la recherche d'une nouvelle résonance massive se désintégrant en une paire électron-positron dans l'état final. De telles résonances
sont prédites par une variété de modèles tels que les théories de grande unification ou les théories introduisant des
dimensions supplémentaires. Leur signature apparaîtrait comme un excès localisé d'événements dans le spectre de masse invariante des paires électron-positron.
La sélection des événements est optimisée pour être très efficace pour les électrons/positron à haute énergie et
éviter de perdre des événements de signal potentiels.
L'analyse repose sur des échantillons simulés pour l'estimation de la principale source de bruit de fond, qui est le processus Drell-Yan du modèle standard.
Des approches basées sur les données sont poursuivies pour la validation de la simulation du bruit de fond avec des électrons/positrons produits dans l'état final et pour la détermination du bruit de fond provenant de processus de \textit{quantum chromodynamics}.
Après avoir inspecté la distribution de masse invariante des paires électron-positron, aucun excès
sur la prédiction du modèle standard n'est observée, et des limites supérieures  à 95\% de niveau de confiance sont placées sur le
rapport entre la section efficace de production multipliée par le rapport de branchement d'une nouvelle résonance et cette même quantité au pic du boson Z, avec les données
prises en 2016 (35.9 \fbinv).
\\La deuxième analyse présentée dans cette thèse est la recherche d'une nouvelle résonance lourde se désintégrant en une paire de photons dans l'état final,
dont l'existence est prédite par des modèles avec des secteurs scalaires non minimaux ou par des théories
qui postulent l'existence de dimensions spatiales supplémentaires.
Leur signature apparaîtrait comme un excès localisé d'événements dans le spectre de masse invariante des paires de photons.
La sélection des événements a été optimisée afin d'être très efficace pour les photons à haute énergie et d'éviter
la perte de signaux potentiels.
L'estimation du bruit de fond est entièrement basée sur les données et réalisée avec une paramétrisation du spectre de masse invariante observé.
Après l'inspection de la masse invariante des paires de photons, aucun excès par rapport à la prédiction du modèle standard n'est observé, et des limites supérieures à 95\% de niveau de confiance sont placées sur la section efficace de production multipliée par le rapport de branchement d'une nouvelle résonance, avec les données
prises jusqu'à Juillet 2016 (12.9 \fbinv).
Les résultats ont également été combinés avec ceux obtenus par les mêmes techniques d'analyse, mais avec des ensembles de données différents enregistrées par l'expérience CMS en 2012 et 2015.
